\newpage
\section{Механизмы защиты}
\setcounter{figure}{0}

При разработке мехнаизма взаимодействия необходимо учесть следующие проблемы:
\begin{itemize}
 \item надежнось передачи данных;
 \item контроль устройств, передающих информацию на ССД;
 \item конфиденциальность передаваемой информации.
\end{itemize}

\subsection{Обеспечение надежной передачи данных} %Для осуществления передачи данных между ССД и УСПД самым простым вариантом является передача данных через сеть интернет. Передача данных по протоколу TCP обеспечит надежность передачи данных, а так же освободит от задачи разработки транспортных механизмов данных в линиях связи.


Под надежностью передачи понимается гарантированная доставка данных от УСПД на ССД. То есть, что все сообщения будут получены без каких либо искажений и потерь. В стеке протоколов TCP/IP на транспортном уровне используется протокол TCP. TCP (англ. Transmission Control Protocol, протокол управления передачей) — один из основных протоколов передачи данных Интернета, предназначенный для управления передачей данных в сетях и подсетях TCP/IP.[4]

Механизм TCP предоставляет поток данных с предварительной установкой соединения, осуществляет повторный запрос данных в случае потери данных и устраняет дублирование при получении двух копий одного пакета, гарантируя целостность передаваемых данных благодаря помехоустойчевому кодированию с возможностью восстановления и уведомление отправителя о результатах передачи.

Использование протокола ТСР на транспортном уровне освобождает от задачи реализации своих механизмов передачи данных и позволяет использовать сети интернет для обмена данными. Это позволит сократить время разработки и способствует решению проблемы с линиями связи между УСПД и ССД. 

При таком подходе будут использоваться линии связи, подведенные в здания интернет-провайдерами, либо выход в сети интернет будет придоставлен при помощи GPRS-модемов. 

\subsection{Контроль подключаемых устройств} %При изучении существующих протоколов было отмечено что в существующих системах УСПД и серверы являлись равнозначными, в разрабатываемой системе предлагается отказаться от данной концепции и сделать УСПД ведомым устройством, то есть отдавать данные только по запросу от ССД. Благодаря такому подходу не будет возможности отправить поддельный пакет на ССД. На УСПД же необходимо хранить идентификационные данные сервера для контроля подключений к УСПД. 

Контроль подключений, является важной задачей ССД, так как нельзя допускать приём данных с несанкционированных устройств. При изучении существующих АСКУЭ было установлено что УСПД само передавало данные на ССД без каких либо запросов со стороны ССД. При таком подходе возможно сформировать ложный пакет и отправить его на ССД. 

Для решения этой проблемы было принято решения применить способ построения системы ``ведущий - ведомый''. Данный способ построения взаимодействия означает что в системе присутствует главное устройство, задачей которого является опрос ``ведомых'' устройств. При таком подходе ССД будет запрашивать данные только от тех устройств, о сушествовании которых ему известно. У данного подхода остается две проблемы:

\begin{itemize}
 \item данный подход не позволяет контролировать подключения к УСПД;
 \item данные подход не защищает от спуфинговых атак.
\end{itemize}

Для решения данной проблемы можно обратиться к технологиям построения защищенного соединения в открытых сетях. 

\subsection{Обеспечение конфиденциальности передаваемых данных} %Для обеспечения конфиденциальности передаваемой информации по откртым сетям необходимо использовать технологии построения виртуальных шифрованных каналов связи, таких как VPN.

Технология построения защищенного соединения в окрытых сетях так же позволит решить проблему конфиденциальности передаваемых данных благодаря тому, что весь передаваемый трафик передается в шифрованном виде. Примером таких технологий является технология VPN. VPN (англ. Virtual Private Network — виртуальная частная сеть[5]) — обобщённое название технологий, позволяющих обеспечить одно или несколько сетевых соединений (логическую сеть) поверх другой сети (например, Интернет). Несмотря на то, что коммуникации осуществляются по сетям с меньшим или неизвестным уровнем доверия (например, по публичным сетям), уровень доверия к построенной логической сети не зависит от уровня доверия к базовым сетям благодаря использованию средств криптографии (шифрования, аутентификации, инфраструктуры открытых ключей, средств для защиты от повторов и изменений передаваемых по логической сети сообщений).

Так же для решения данной задачи возможно использование технологии SSH. SSH (англ. Secure Shell — «безопасная оболочка»[6]) — сетевой протокол прикладного уровня, позволяющий производить удалённое управление операционной системой и туннелирование TCP-соединений (например, для передачи файлов). Схож по функциональности с протоколами Telnet и rlogin, но, в отличие от них, шифрует весь трафик, включая и передаваемые пароли. SSH допускает выбор различных алгоритмов шифрования. SSH-клиенты и SSH-серверы доступны для большинства сетевых операционных систем. 

%TODO тут можно ещё чего-нибудь накорябать