\newpage
\section{Механизмы защиты}
\setcounter{figure}{0}

При разработке мехнаизма взаимодействия необходимо учесть следующие проблемы:
\begin{itemize}
 \item надежнось передачи данных;
 \item контроль устройств, передающих информацию на ССД;
 \item конфиденциальность передаваемой информации.
\end{itemize}

\subsection{Обеспечение надежной передачи данных} %Для осуществления передачи данных между ССД и УСПД самым простым вариантом является передача данных через сеть интернет. Передача данных по протоколу TCP обеспечит надежность передачи данных, а так же освободит от задачи разработки транспортных механизмов данных в линиях связи.


Под надежностью передачи понимается гарантированная доставка данных от УСПД на ССД. То есть, что все сообщения будут получены без каких либо искажений и потерь. В стеке протоколов TCP/IP на транспортном уровне используется протокол TCP. TCP (англ. Transmission Control Protocol, протокол управления передачей) — один из основных протоколов передачи данных Интернета, предназначенный для управления передачей данных в сетях и подсетях TCP/IP.[4]

Механизм TCP предоставляет поток данных с предварительной установкой соединения, осуществляет повторный запрос данных в случае потери данных и устраняет дублирование при получении двух копий одного пакета, гарантируя целостность передаваемых данных благодаря помехоустойчевому кодированию с возможностью восстановления и уведомление отправителя о результатах передачи.

Использование протокола ТСР на транспортном уровне освобождает от задачи реализации своих механизмов передачи данных и позволяет использовать сети интернет для обмена данными. Это позволит сократить время разработки и способствует решению проблемы с линиями связи между УСПД и ССД. 

При таком подходе будут использоваться линии связи, подведенные в здания интернет-провайдерами, либо выход в сети интернет будет придоставлен при помощи GPRS-модемов. 

\subsection{Контроль подключаемых устройств} %При изучении существующих протоколов было отмечено что в существующих системах УСПД и серверы являлись равнозначными, в разрабатываемой системе предлагается отказаться от данной концепции и сделать УСПД ведомым устройством, то есть отдавать данные только по запросу от ССД. Благодаря такому подходу не будет возможности отправить поддельный пакет на ССД. На УСПД же необходимо хранить идентификационные данные сервера для контроля подключений к УСПД. 

\subsection{Обеспечение конфиденциальности передаваемых данных} %Для обеспечения конфиденциальности передаваемой информации по откртым сетям необходимо использовать технологии построения виртуальных шифрованных каналов связи, таких как VPN.


