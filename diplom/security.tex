\newpage
\section{Механизмы защиты}
\setcounter{figure}{0}

При разработке мехнаизма взаимодействия необходимо учесть следующие проблемы:
\begin{itemize}
 \item надежнось передачи данных;
 \item контроль устройств, передающих информацию на ССД;
 \item конфиденциальность передаваемой информации.
\end{itemize}

Для осуществления передачи данных между ССД и УСПД самым простым вариантом является передача данных через сеть интернет. Передача данных по протоколу TCP обеспечит надежность передачи данных, а так же освободит от задачи разработки транспортных механизмов данных в линиях связи.

При изучении существующих протоколов было отмечено что в существующих системах УСПД и серверы являлись равнозначными, в разрабатываемой системе предлагается отказаться от данной концепции и сделать УСПД ведомым устройством, то есть отдавать данные только по запросу от ССД. Благодаря такому подходу не будет возможности отправить поддельный пакет на ССД. На УСПД же необходимо хранить идентификационные данные сервера для контроля подключений к УСПД. 

Для обеспечения конфиденциальности передаваемой информации по откртым сетям необходимо использовать технологии построения виртуальных шифрованных каналов связи, таких как VPN.

%TODO тоже как то маловато =(