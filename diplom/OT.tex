\section{Разработка вопросов охраны труда}

В соответствии с ГОСТ 12.0.002-80 рассматриваются следующие вопросы, касающиеся безопасности жизнедеятельности:

\begin{enumerate}
 \item Анализ опасных и вредных производственных факторов.
 \item Требования безопасности и комплекс защитных мероприятий (с указанием НТД – ГОСТы, СанПиНы, НРБ).
 \item Инструкции по технике безопасности (перед началом работы, во время работы, по окончанию работы, в аварийных ситуациях, ЗАПРЕЩЕНО).
 \item Ответственность при нарушении правил ОТ. 
\end{enumerate}

\subsection{Анализ опасных и вредных производственных факторов}

В соответствии с ГОСТ 12.0.003-74 ССБТ «Система стандартов безопасности труда. Опасные и вредные производственные факторы» выделяют следующие факторы:

\begin{enumerate}
 \item Повышенный уровень шума, вибрации;
 \item Отсутствие или недостаток освещения;
 \item Микроклимат;
 \item Повышенные уровни статического электричества, электромагнитных излучений и другие [7].
\end{enumerate}

\subsection{Требования безопасности}

\subsubsection{Требования к освещению на рабочих местах}

Для рабочих мест, оснащенных терминалами, рекомендуется освещённость 300 – 500 Лк. Общего освещения недостаточно, поэтому необходимо применять местное искусственное освещение.

Согласно санитарно-гигиеническим требованиям рабочее место (РМ) инженера должна освещаться естественным и искусственным освещением.

Освещённость помещения должна отвечать следующим требованиям:

\begin{itemize}
 \item уровень освещённости рабочего места должен соответствовать яркости монитора ЭВМ;
 \item распределение яркости на рабочей поверхности и в окружающем пространстве должно быть достаточно равномерными, на рабочей поверхности не должно быть резких теней;
 \item величина освещённости должна быть постоянной во времени.
\end{itemize}

Кроме того, необходимо обеспечить долговечность, экономичность, электро- и пожаробезопасность, эстетичность и простоту эксплуатации осветительных приборов.

По нормам освещенности СНИП 23-05-95 и отраслевым нормам, работа инженера относится к четвертому разряду зрительной работы. Для этого разряда рекомендуется освещенность 300 Лк. 

Требуемая площадь светового проема при боковом естественном освещении, при площади помещения в 10 м\textsuperscript{2}, составляет 7м\textsuperscript{2}. Учитывая, что в помещении площадь оконного проема составляет около 4 м\textsuperscript{2}, применение лишь одного бокового освещения недостаточно. Следовательно, в помещении необходимо использовать искусственное освещение. 

Согласно норме, освещенность Е должна быть равна 300 – 500 Лк. 

После подсчетов светового потока в помещении, освещенность Е = 390 Лк. Таким образом расчет показывает, что освещенность рабочего помещения соответствует СНИП 23-05-95. И Дополнительное освещение не требуется.

\subsubsection{Требования к микроклимату}

\begin{enumerate}
 \item В помещениях, оборудованных ПЭВМ, должна проводиться ежедневная влажная уборка и систематическое проветривание после каждого часа работы на ПЭВМ.
 \item Содержание вредных химических веществ в помещениях, в которых работа с использованием ПЭВМ является основной, не должно превышать предельно допустимых концентраций загрязняющих веществ в атмосферном воздухе населенных мест в соответствии с действующими гигиеническими нормативами [8].
\end{enumerate}

\subsubsection{Требования к уровню шума}

\begin{enumerate}
 \item 1. В производственных помещениях при выполнении основных или вспомогательных работ с использованием ПЭВМ уровни шума на рабочих местах не должны превышать предельно допустимых значений, установленных для данных видов работ в соответствии с действующими санитарно-эпидемиологическими нормативами.
 \item 2. При выполнении работ с использованием ПЭВМ в производственных помещениях уровень вибрации не должен превышать допустимых значений вибрации для рабочих мест в соответствии с действующими санитарно-эпидемиологическими нормативами.
\end{enumerate}

Допустимый уровень шума на рабочем месте составляет 60 дБ.

Шум, создаваемый ЭВМ является постоянным и составляет 5 дБ. Шум, производимый принтером – непостоянный. Таким образом, на рабочем месте инженера превышения звукового давления не наблюдается.

\subsubsection{Требования к организации и оборудованию рабочих мест ПЭВМ}

Для уменьшения влияния на работу психофизиологических ОВПФ рабочее место, при выполнении действий в положении сидя должно соответствовать нормам ГОСТ 12.2.032-78.

%TODO INSERT TABLE