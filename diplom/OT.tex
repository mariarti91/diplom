\section{Разработка вопросов охраны труда}

В соответствии с ГОСТ 12.0.002-80 рассматриваются следующие вопросы, касающиеся безопасности жизнедеятельности:

\begin{enumerate}
 \item Анализ опасных и вредных производственных факторов.
 \item Требования безопасности и комплекс защитных мероприятий (с указанием НТД – ГОСТы, СанПиНы, НРБ).
 \item Инструкции по технике безопасности (перед началом работы, во время работы, по окончанию работы, в аварийных ситуациях, ЗАПРЕЩЕНО).
 \item Ответственность при нарушении правил ОТ. 
\end{enumerate}

\subsection{Анализ опасных и вредных производственных факторов}

В соответствии с ГОСТ 12.0.003-74 ССБТ «Система стандартов безопасности труда. Опасные и вредные производственные факторы» выделяют следующие факторы:

\begin{enumerate}
 \item Повышенный уровень шума, вибрации;
 \item Отсутствие или недостаток освещения;
 \item Микроклимат;
 \item Повышенные уровни статического электричества, электромагнитных излучений и другие [7].
\end{enumerate}

\subsection{Требования безопасности}

\subsubsection{Требования к освещению на рабочих местах}

Для рабочих мест, оснащенных терминалами, рекомендуется освещённость 300 – 500 Лк. Общего освещения недостаточно, поэтому необходимо применять местное искусственное освещение.

Согласно санитарно-гигиеническим требованиям рабочее место (РМ) инженера должна освещаться естественным и искусственным освещением.

Освещённость помещения должна отвечать следующим требованиям:

\begin{itemize}
 \item уровень освещённости рабочего места должен соответствовать яркости монитора ЭВМ;
 \item распределение яркости на рабочей поверхности и в окружающем пространстве должно быть достаточно равномерными, на рабочей поверхности не должно быть резких теней;
 \item величина освещённости должна быть постоянной во времени.
\end{itemize}

Кроме того, необходимо обеспечить долговечность, экономичность, электро- и пожаробезопасность, эстетичность и простоту эксплуатации осветительных приборов.

По нормам освещенности СНИП 23-05-95 и отраслевым нормам, работа инженера относится к четвертому разряду зрительной работы. Для этого разряда рекомендуется освещенность 300 Лк. 

Требуемая площадь светового проема при боковом естественном освещении, при площади помещения в 10 м\textsuperscript{2}, составляет 7м\textsuperscript{2}. Учитывая, что в помещении площадь оконного проема составляет около 4 м\textsuperscript{2}, применение лишь одного бокового освещения недостаточно. Следовательно, в помещении необходимо использовать искусственное освещение. 

Согласно норме, освещенность Е должна быть равна 300 – 500 Лк. 

После подсчетов светового потока в помещении, освещенность Е = 390 Лк. Таким образом расчет показывает, что освещенность рабочего помещения соответствует СНИП 23-05-95. И Дополнительное освещение не требуется.

\subsubsection{Требования к микроклимату}

\begin{enumerate}
 \item В помещениях, оборудованных ПЭВМ, должна проводиться ежедневная влажная уборка и систематическое проветривание после каждого часа работы на ПЭВМ.
 \item Содержание вредных химических веществ в помещениях, в которых работа с использованием ПЭВМ является основной, не должно превышать предельно допустимых концентраций загрязняющих веществ в атмосферном воздухе населенных мест в соответствии с действующими гигиеническими нормативами [8].
\end{enumerate}

\subsubsection{Требования к уровню шума}

\begin{enumerate}
 \item 1. В производственных помещениях при выполнении основных или вспомогательных работ с использованием ПЭВМ уровни шума на рабочих местах не должны превышать предельно допустимых значений, установленных для данных видов работ в соответствии с действующими санитарно-эпидемиологическими нормативами.
 \item 2. При выполнении работ с использованием ПЭВМ в производственных помещениях уровень вибрации не должен превышать допустимых значений вибрации для рабочих мест в соответствии с действующими санитарно-эпидемиологическими нормативами.
\end{enumerate}

Допустимый уровень шума на рабочем месте составляет 60 дБ.

Шум, создаваемый ЭВМ является постоянным и составляет 5 дБ. Шум, производимый принтером – непостоянный. Таким образом, на рабочем месте инженера превышения звукового давления не наблюдается.

\subsubsection{Требования к организации и оборудованию рабочих мест ПЭВМ}

Для уменьшения влияния на работу психофизиологических ОВПФ рабочее место, при выполнении действий в положении сидя должно соответствовать нормам ГОСТ 12.2.032-78.

\begin{center}
 \begin{longtable}[h]{|*3{m{0.3\textwidth}|}}
  \caption{Нормативы и результаты рабочих мест ПЭВМ} \label{tab:pevm} \\
  \hline
  Наименование & Нормативы & Реальные результаты \\
  \hline
  \multicolumn{3}{|c|}{Стол} \\
  \hline  
  Высота рабочей поверхности стола & 68-80см & 75см \\
  \hline
  Пространство для ног & Высота не менее 60см, Ширина не менее 50см & Высота 67см, Ширина 54см \\
  \hline
  Глубина на уровне колен & Не менее 45см & 54см \\
  \hline
  На уровне вытянутых ног & Не менее 65см & 70см \\
  \hline
  \multicolumn{3}{|c|}{Стул} \\
  \hline
  Ширина и глубина поверхности сидения & Не менее 40см & 44см\\
  \hline
  Регулировка высоты поверхности сидения & 40-55см & 40-53см\\
  \hline
  Регулировка углов наклона & Вперед до 15 градусов, Назад до 5 градусов & Вперед до 12-14 градусов, Назад до 4-5 градусов \\
  \hline
  Высота опорной поверхности спинки & 1-5см & 5см \\
  \hline
  Ширина опорной поверхности спинки & Не менее 38см & 47см \\
  \hline
  Угол наклона спинки в вертикальной плоскости & $\pm$ 30 градусов & $\pm$ 30 градусов \\
  \hline
  Стационарные или съемные подлокотники & Длина не менее 25см, Ширина 5-7см & Длина 41см, Ширина 6см \\
  \hline
 \end{longtable}
\end{center}

\subsubsection{Требования пожарной безопасности}

Работающие в помещении люди обязаны знать и соблюдать правила пожарной безопасности.

Помещение должно быть оснащено средствами пожаротушения в соответствии с ГОСТ 12.1.004 - 91 «Пожарная безопасность. Общие требования».

В помещении запрещается:

\begin{itemize}
 \item оставлять без присмотра электроустановки и электронагревательные приборы;
 \item сушить горючие предметы на отопительных приборах;
 \item Для предотвращения пожара и ряда его последствий проводятся следующие организационно-технические мероприятия:
 \item составление плана эвакуации и действий персонала при пожаре;
 \item проведение инструктажа пожарной безопасности;
 \item назначение лиц, ответственных за пожарную безопасность;
 \item размещение первичных средств пожаротушения;
 \item установка пожарной сигнализации;
 \item контроль температуры оборудования.
\end{itemize}

План эвакуации при пожаре создаётся по параметрам ГОСТ СНиП 2.01.02-85.

В случае возникновения пожара следует немедленно:

\begin{itemize}
 \item обесточить помещение с помощью рубильника на силовом щите;
 \item сообщить о пожаре по телефону 01;
 \item удалить из опасной зоны всех не занятых в ликвидации пожара людей;
 \item до прибытия пожарных самостоятельно приступить к ликвидации пожара имеющимися в наличии средствами пожаротушения.
\end{itemize}

Соблюдение всех вышеперечисленных требований позволяет повысить производительность труда персонала, способствует снижению количества ошибок в работе и случаев травматизма, а также благоприятно сказывается на объективности управленческих решений [7].

\subsubsection{Меры защиты от статического электричества}

Основным способом предупреждения возникновения электростатического заряда является постоянный отвод статического электричества от технологического оборудования с помощью заземления. 

Для непрерывного снятия электростатических зарядов с человека используются электропроводящие полы, заземленные зоны или рабочие площадки, оборудование, трапы.

\subsection{Инструкции по технике безопасности}

\subsubsection{Инструкция безопасности перед началом работы}

\begin{enumerate}
 \item Подготовить рабочее место.
 \item Отрегулировать освещение на рабочем месте, убедиться в отсутствии бликов на экране.
 \item Проверить правильность подключения оборудования к электросети.
 \item Проверить исправность проводов питания и отсутствие оголенных участков проводов.
 \item Убедиться в наличии заземления системного блока, монитора и защитного экрана.
 \item Протереть антистатической салфеткой поверхность экрана монитора и защитного экрана.
 \item Проверить правильность установки стола, стула, подставки для ног, при необходимости произвести регулировку рабочего стола и кресла, а также расположение элементов компьютера в соответствии с требованиями эргономики и в целях исключения неудобных поз и длительных напряжений тела [9].
\end{enumerate}

\subsubsection{Инструкция безопасности во время работы}

\begin{enumerate}
 \item Продолжительность непрерывной работы с компьютером без регламентированного перерыва не должна превышать 2-х часов.
 \item В целях обеспечения защиты от электромагнитных и электростатических полей допускается применение приэкранных фильтров и специальных экранов, прошедших испытания в аккредитованных лабораториях и имеющих гигиенический сертификат.
 \item Не оставлять без присмотра включенные ПЭВМ и отдельные устройства.
 \item Не производить перекомплектацию ПЭВМ без представителя технической сервисной службы.
 \item Перед использованием проверять съемные носители информации антивирусными программами компьютера на наличие вредоносных программ.
 \item Не устанавливать неизвестные системы паролирования и самостоятельно проводить переформатирование диска [9].
\end{enumerate}

\subsubsection{Инструкция безопасности в аварийных ситуациях}

\begin{enumerate}
 \item Во всех случаях обрыва проводов питания, неисправности заземления и других повреждений, появления гари, немедленно отключить питание и сообщить об аварийной ситуации руководителю.
 \item Не приступать к работе до устранения неисправностей.
 \item При получении травм или внезапном заболевании немедленно известить своего руководителя, организовать первую доврачебную помощь или вызвать скорую медицинскую помощь [9].
\end{enumerate}

\subsubsection{Инструкция безопасности по окончании работы}

\begin{enumerate}
 \item Отключить ПК от сети, штепсельную вилку при этом держать за корпус. Запрещается отключать ПК за электропровод. При отключении ПК со съемным шнуром питания сначала необходимо отключить вилку от розетки, а затем отключить питающий шнур от ПК.
 \item Привести в порядок рабочее место.
 \item Выполнить упражнения для глаз и пальцев рук на расслабление [9].
\end{enumerate}

\subsubsection{ЗАПРЕЩЕНО}

\begin{enumerate}
 \item Прикасаться к задней панели системного блока при включенном питании;
 \item Переключать разъемы интерфейсных кабелей периферийных устройств при включенном питании;
 \item Допускать попадание влаги на поверхность системного блока (процессора), монитора, рабочую поверхность клавиатуры, дисководов, принтеров и других устройств;
 \item Производить самостоятельное вскрытие и ремонт оборудования;
 \item Работать на компьютере при снятых кожухах;
 \item Отключать оборудование от электросети и выдергивать электровилку, держась за шнур [9].
\end{enumerate}
