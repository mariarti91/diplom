\section{Угрозы информации, передаваемой между УСПД и ССД}

Информация, передаваемая от сервера сбора данных(ССД) к устройству сбора и передачи данных(УСПД) - это команды, которые обрабатывает УСПД. Информация, передаваемая от УСПД к ССД это показания устройств учета(УУ), статус УУ, статус УСПД. Передача информации происходит по глобальной сети интернет. Каналами может выступать проводное (ethernet) и беспроводное (GPRS) соединение.

\subsection{Виды угроз}

\begin{enumerate}
 \item угрозы нарушения физической целостности:
 \begin{itemize}
  \item разрушение каналов связи;
 \end{itemize}
 
 \item угрозы логической структуры:
 \begin{itemize}
  \item передача некорректных данных;
  \item ошибки адресации;
 \end{itemize}
 
 \item угрозы нарушения содержания:
 \begin{itemize}
  \item изменение передаваемой информации;
  \item получение поддельного пакета;
 \end{itemize}
 
 \item угрозы нарушения конфиденциальности:
 \begin{itemize}
  \item перехват идентификационных данных;
  \item перехват показаний пользователей;
 \end{itemize}
 
 \item угрозы нарушения прав собственности на информацию:
 \begin{itemize}
  \item подмена показаний;
  \item подмена идентификаторов УУ;
 \end{itemize}
 
\end{enumerate}

\subsection{Характеры возникновения угроз}

\begin{center}
 \begin{longtable}[h!]{|*2{p{0.4\linewidth}|}}
  \caption{Классификация угроз} \label{tab:threats_group} \\
  \hline
  Естественные& Умышленные \\
  \hline
 \end{longtable}

\end{center}
