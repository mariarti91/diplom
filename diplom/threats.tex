\newpage
\section{Угрозы информации, передаваемой между УСПД и ССД}
\setcounter{figure}{0}

Информация, передаваемая от сервера сбора данных(ССД) к устройству сбора и передачи данных(УСПД) - это команды, которые обрабатывает УСПД. Информация, передаваемая от УСПД к ССД это показания устройств учета(УУ), статус УУ, статус УСПД. Передача информации происходит по глобальной сети интернет. Каналами может выступать проводное (ethernet) и беспроводное (GPRS) соединение.

\subsection{Виды угроз}

\begin{enumerate}
%  \item угрозы нарушения физической целостности:
%  \begin{itemize}
%   \item разрушение каналов связи;
%  \end{itemize}
% TODO узнать про данный раздел НЕ ЗАБЫТЬ ДОБАВИТЬ В ТАБЛИЦУ!!!!

 \item угрозы логической структуры:
 \begin{itemize}
  \item передача некорректных данных;
  \item ошибки адресации;
 \end{itemize}
 
 \item угрозы нарушения содержания:
 \begin{itemize}
  \item изменение передаваемой информации;
  \item получение поддельного пакета;
 \end{itemize}
 
 \item угрозы нарушения конфиденциальности:
 \begin{itemize}
  \item перехват идентификационных данных;
  \item перехват показаний пользователей;
 \end{itemize}
 
 \item угрозы нарушения прав собственности на информацию:
 \begin{itemize}
  \item подмена показаний;
  \item подмена идентификаторов УУ;
 \end{itemize}
 
\end{enumerate}

\subsection{Характеры возникновения угроз}

В таблице \ref{tab:threats_group} приведена классификация угроз по природе их возникновения.

\begin{center}
 \begin{longtable}[h!]{|*2{p{0.4\linewidth}|}}
  \caption{Классификация угроз} \label{tab:threats_group} \\
  \hline
  Естественные & Умышленные \\
  \hline
  передача некорректных данных & передача некорректных данных\\ 
  \hline
  ошибки адресации & изменение передаваемой информации\\
  \hline
  & получение поддельного пакета\\
  \hline
  & перехват идентификационных данных\\
  \hline
  & перехват показаний пользователей\\
  \hline
  & подмена показаний\\
  \hline
  & подмена идентификаторов УУ\\
  \hline
  
 \end{longtable}

\end{center}

\subsection{Источники угроз}

\begin{enumerate}
 \item Люди:
 \begin{itemize}
  \item сервисный инженер;
  \item пользователи системы;
  \item работники ЖКУ;
 \end{itemize}
 
 \item Технические средства:
 \begin{itemize}
  \item устройства съема ПЭМИ;
  \item вредоносное ПО;
 \end{itemize}

 \item Прочие источники:
 \begin{itemize}
  \item грызуны.
 \end{itemize}

\end{enumerate}

\subsection{Предпосылки угроз}

\begin{enumerate}
%  \item угрозы нарушения физической целостности:
%  \begin{itemize}
%   \item разрушение каналов связи;
%  \end{itemize}
% TODO узнать про данный раздел НЕ ЗАБЫТЬ ДОБАВИТЬ В ТАБЛИЦУ!!!!

 \item угрозы логической структуры:
 \begin{itemize}
  \item передача некорректных данных \\ искажение информации при передаче;
  \item ошибки адресации \\ появление у злоумышленника идеи спуфинговых атак;
 \end{itemize}
 
 \item угрозы нарушения содержания:
 \begin{itemize}
  \item изменение передаваемой информации \\ желание клиента уменьшить платы за энергоресурсы;
  \item получение поддельного пакета \\ желание навредить клиенту;
 \end{itemize}
 
 \item угрозы нарушения конфиденциальности:
 \begin{itemize}
  \item перехват идентификационных данных \\ желание использовать учетную запись инженера для искажения настроке счетчиков или УСПД;
  \item перехват показаний пользователей \\ желание узнать количество потребляемых энергоресурсов;
 \end{itemize}
 
 \item угрозы нарушения прав собственности на информацию:
 \begin{itemize}
  \item подмена показаний \\ желание клиента уменьшить платы за энергоресурсы;
  \item подмена идентификаторов УУ \\ желание вывести из стороя УУ;
 \end{itemize}
 
\end{enumerate}

\subsection{Каналы реализации угроз}

\begin{itemize}
 \item подключение к каналам связи;
 \item выведывание информации у персонала;
 \item подключение несанкционированных устройсв;
 \item воздействие ЭМИ.
\end{itemize}

%TODO помоему это не то =(