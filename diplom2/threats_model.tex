\newpage
\section{Угрозы}
\setcounter{figure}{0}

Согласно выписке из базовой модели угроз ФСТЭК России 2008 года \cite{threats} при обраотке информации в распределенных информационных системах, имеющих подключение к сетям общего пользования и(или) сетям международного информационного обмена, возможна реализация следующих угроз:

\begin{itemize}
 \item угроза утечки информации по техническим каналам;
 \item угроза несанкционированного доступа к информации, обрабатываемой на автоматизированном рабочем месте.
\end{itemize}

\subsection{Угрозы утечки информации по техническим каналам}

Угрозы утечки информации по техническим каналам включают в себя:

\begin{itemize}
 \item угрозы утечки акустической информации;
 \item угрозы утечки видовой информации;
 \item угрозы утечки информации по каналам ПЭМИН.
\end{itemize}

Так как рассматривается только протокол обмена данными, угрозы утечки акустической и видовой информации не актуальны.

\subsection{Угрозы несанкционированного доступа к информации}

Угрозы НСД, связанные с действиями нарушителей, имеющих доступ к ИС включает в себя угрозы из трех моделей:

\begin{itemize}
 \item Модель угроз для АРМ без подключения к общедоступным сетям;
 \item Модель угроз для распределенных систем без подключения к сетям;
 \item Модель угроз для распределенных систем с подключением к общедоступным сетям.
\end{itemize}

Так как в данной работе рассматривается только протокол взаимодействия, а не система в целом. Угрозы реалезуемые непосредственно на АРМ рассматривать не будут. Тогда перечень угроз НСД для данной части системы:

\begin{itemize}
 %тут угрозы из модели без доступа к сети. тоесть тупо внутренняя сеть системы.
 \item угрозы ``Анализа сетевого трафика'' с перехватом передаваемой по сети информации;
 \item грозы сканирования, направленные на выявление открытых портов и служб, открытых соединение и др.;
 \item угрозы внедрения ложного объекта сети;
 \item угрозы навязывания ложного маршрута путем несанкционированного изменения маршрутно-адресных данных;
 \item угрозы выявления паролей;
 \item угрозы типа ``отказ в обслуживании'';
 \item угрозы удаленного запуска приложений;
 \item угрозы внедрения по сети вредоносных программ;
 %а вот тут уже угрозы из внешней сети
 \item угрозы ``Анализа  сетевого  трафика'' с  перехватом  передаваемой из ИС и принимаемой в ИС из внешних сетей информации;
 \item угрозы  сканирования,  направленные  на  выявление типа  или  типов используемых операционных систем, сетевых адресов рабочих станций ИС, топологии сети, открытых портов и служб, открытых соединений и др.;
 \item угрозы внедрения ложного объекта как в ИС, так и во внешних сетях;
 \item угрозы подмены доверенного объекта;
 \item угрозы  навязывания  ложного  маршрута  путем  несанкционированного изменения  маршрутно-адресных  данных  как  внутри сети,  так  и  во внешних сетях;
 \item угрозы выявления паролей;
 \item угрозы типа ``Отказ в обслуживании'';
 \item угрозы удаленного запуска приложений;
 \item угрозы внедрения по сети вредоносных программ.
\end{itemize}

\subsection{Угрозы информации передаваемой между УСПД и ССД}

В данной части информационной системы актуальны следующие угрозы:

\begin{itemize}
 \item угрозы утечки информации по каналам ПЭМИН;
 \item угрозы внедрения ложного объекта сети;
 \item угрозы ``Анализа сетевого трафика'' с перехватом передаваемой из ИС и принимаемой в ИС из внешних сетей информации;
 \item угрозы подмены доверенного объекта;
 \item угрозы навязывания ложного маршрута путем несанкционированного изменения маршрутно-адресных во внешних сетях;
 \item угрозы выявления паролей.
\end{itemize}

В таблице \ref{tab:threats} указаны возможные последствия реализации угроз.

\begin{table}[!ht]
 \caption{Возможные последствия реализации угрз} 
 \label{tab:threats}
%  \begin{center}
  \begin{tabular}{|p{0.35\linewidth}|p{0.55\linewidth}|}
   \hline
   Тип атаки & Возможные последствия \\ 
   \hline
   Съем ПЭМИН & Перехват   передаваемых   данных,  в том числе идентификаторов и паролей пользователей \\
   \hline
   Внедрение ложного объекта сети & Перехват и просмотр трафика. Несанкционированный доступ  к  сетевым  ресурсам, навязывание ложной информации \\
   \hline
   Анализ сетевого трафика & Исследование   характеристик   сетевого   трафика, перехват   передаваемых   данных,  в том числе идентификаторов и паролей пользователей\\
   \hline
   Подмена доверенного объекта & Изменение    трассы    прохождения    сообщений, несанкционированное    изменение   маршрутно-адресных  данных.  Несанкционированный  доступ  к сетевым ресурсам, навязывание ложной информации\\
   \hline
   Навязывание ложного маршрута & Несанкционированное    изменение    маршрутно-адресных    данных,    анализ    и    модификация передаваемых    данных,   навязывание    ложных сообщений\\
   \hline
   Выявления паролей & Выполнение   любого   деструктивного   действия, связанного  с  получением  несанкционированного доступа\\
   \hline
  \end{tabular}
%  \end{center}
\end{table}