\newpage
\section{Введение}
\setcounter{figure}{0}

%Целью данной работы является определить наиболее популярные протоколы передачи данных, поддерживаемые в приборах и системах учета, с целью их анализа, выбора и дальнейшей реализации в разрабатываемой автоматизированной системе коммерческого учета энергоресурсов, 

Целью данной работы является создание протокола взаимодействия между сервером сбора данных (ССД) и устройствами сбора и передачи данных(УСПД) в автоматизированной системе коммерческого учета энергоресурсов (АСКУЭ).

Создание нового протокола необходимо для использования АСКУЭ в сфере ЖКУ, так как существующие протоколы ориентированы на использования в пределах предприятий, из-за чего не учитывают аспекты информационной безопасности. В протоколах отсуствует авторизация, линии связи обладают недостаточно дальностью передачи сигнала без использования повторителей. 

Всё это приводит к серьезным проблемам при использовании АСКУЭ в сфере ЖКУ. Появляется проблема передачи данных от УСПД до ССД, так как для использования существующих протоколов необходимо прокладывать линии связи. Использование протоколов так же не гарантирует подлинности и аутентичность передаваемых данных. 

%После чего необходимо реализовать функции взаимодействия сервера с устройством съема и передачи данных (УСПД) по данному протоколу, в том числе разработать систему команд взаимодействия с УСПД.