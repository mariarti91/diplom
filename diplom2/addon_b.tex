\ESKDappendix{Справочное}{\normalfont Исходные коды эмулятора ССД \label{app:code_ssd}}

main.cpp

\begin{lstlisting}
#include "mainwindow.h"
#include <QApplication>

int main(int argc, char *argv[])
{
    QApplication a(argc, argv);
    MainWindow w;
    w.show();
    return a.exec();
}
\end{lstlisting}

mainwindow.h:

\begin{lstlisting}
#ifndef MAINWINDOW_H
#define MAINWINDOW_H

#include <QMainWindow>
#include <QStandardItemModel>

#include "myserver.h"
#include "adduserdialog.h"

namespace Ui {
class MainWindow;
}

class MainWindow : public QMainWindow
{
    Q_OBJECT

public:
    explicit MainWindow(QWidget *parent = 0);
    ~MainWindow();

private:
    Ui::MainWindow *ui;
    MyServer server;
public slots:
    void slotAddClientPbCliced();
};

#endif // MAINWINDOW_H
\end{lstlisting}

mainwindow.cpp:

\begin{lstlisting}
#include "mainwindow.h"
#include "ui_mainwindow.h"

MainWindow::MainWindow(QWidget *parent) :
    QMainWindow(parent),
    ui(new Ui::MainWindow)
{
    ui->setupUi(this);
}
//--------------------------------------------------------------------------

MainWindow::~MainWindow()
{
    delete ui;
}
//--------------------------------------------------------------------------

void MainWindow::slotAddClientPbCliced()
{
    AddUserDialog d(this);
    if(d.exec())
    {
        server.addClient(d.getClientName(), d.getClientAddress(), d.getClientPort());

        QMap<QString,ClientAddress>* buf = server.getClientsList();
        QStandardItemModel *model = new QStandardItemModel(buf->count(), 3, ui->tvClients);
        ui->tvClients->setModel(model);
        ui->teClients->clear();
        int rowCount = 0;
        foreach (QString client, buf->keys())
        {
            QStandardItem* name = new QStandardItem(client);
            QStandardItem* addr = new QStandardItem(buf->value(client).addr);
            QStandardItem* port = new QStandardItem(tr("%1").arg(buf->value(client).port));
            model->setItem(rowCount, 0, name);
            model->setItem(rowCount, 1, addr);
            model->setItem(rowCount, 2, port);
            rowCount++;
        }
    }
    else
    {
        //ui->teClients->setText("CANCEL");
    }
}
//--------------------------------------------------------------------------
\end{lstlisting}

myserver.h:

\begin{lstlisting}
#ifndef MYSERVER_H
#define MYSERVER_H

#include <QObject>
#include <QString>
#include <QMap>
#include <QTcpServer>
#include <QTimerEvent>

#include "clientsock.h"

struct ClientAddress
{
    ClientAddress();
    ClientAddress(const QString&, const int&);
    ~ClientAddress();
    QString addr;
    int port;
};

class MyServer : public QObject
{
    Q_OBJECT
public:
    explicit MyServer(QObject *parent = 0);
    ~MyServer();

    void addClient(const QString &name, const QString &addr, const int &port);
    QMap<QString, ClientAddress>* getClientsList();

private:
    QMap<QString, ClientAddress> clients;
    void timerEvent(QTimerEvent* event);

signals:

public slots:

};

#endif // MYSERVER_H
\end{lstlisting}

myserver.cpp:

\begin{lstlisting}
#include "myserver.h"

ClientAddress::ClientAddress(){}
//----------------------------------------------------------------------------

ClientAddress::ClientAddress(const QString & a, const int & p)
{
    this->addr = a;
    this->port = p;
}
//----------------------------------------------------------------------------

ClientAddress::~ClientAddress(){}
//============================================================================

MyServer::MyServer(QObject *parent) :
    QObject(parent)
{
    startTimer(10000);
}
//----------------------------------------------------------------------------

MyServer::~MyServer(){}
//----------------------------------------------------------------------------

void MyServer::addClient(const QString &name,
                         const QString &addr,
                         const int &port)
{
    clients.insert(name, ClientAddress(addr, port));
}
//----------------------------------------------------------------------------

QMap<QString, ClientAddress>* MyServer::getClientsList()
{
    return &clients;
}
//----------------------------------------------------------------------------

void MyServer::timerEvent(QTimerEvent *event)
{
    std::cout << "event!" << std::endl;
    foreach (QString name, this->clients.keys())
    {
        //ClientSock sc(clients.value(name).addr, clients.value(name).port);
        //sc.setName(name);
        //sc.sendData("test");
        ClientSock* sc = new ClientSock(clients.value(name).addr, clients.value(name).port);
        sc->setName(name);
        sc->sendData("get_request");
    }
}
//----------------------------------------------------------------------------
\end{lstlisting}

clientsock.h:

\begin{lstlisting}
#ifndef CLIENTSOCK_H
#define CLIENTSOCK_H

#include <QObject>
#include <QString>
#include <QTcpSocket>
#include <QTimer>
#include <QTimerEvent>
#include <iostream>

class ClientSock : public QObject
{
    Q_OBJECT
public:
    explicit ClientSock(QObject *parent = 0);
    explicit ClientSock(QString addr, int port, QObject *parent = 0);
    explicit ClientSock(QTcpSocket *sock, QObject *parent = 0);
    ~ClientSock();
    void initSc(const QString & addr, const int & port);
    void setName(const QString & name);
    QString getName();
    void sendData(QByteArray data);
    void timerEvent(QTimerEvent *event);
private:
    QTcpSocket * sc;
    QString m_qsScName;
signals:
    void scDisconnect(QString);
public slots:
    QByteArray getData();
};

#endif // CLIENTSOCK_H
\end{lstlisting}

clientsock.cpp:

\begin{lstlisting}
#include "clientsock.h"

ClientSock::ClientSock(QObject *parent) :
    QObject(parent)
{
    sc = new QTcpSocket(this);
}
//--------------------------------------------------------------------

ClientSock::ClientSock(QString addr, int port, QObject *parent) :
    QObject(parent)
{
    sc = new QTcpSocket(this);
    initSc(addr, port);
}
//--------------------------------------------------------------------

ClientSock::ClientSock(QTcpSocket *sock, QObject *parent) :
    QObject(parent)
{
    sc = sock;
    this->setName(QString("%1").arg(sock->peerPort()));
    connect(sc, SIGNAL(readyRead()), this, SLOT(getData()));
    startTimer(5000);
}

//--------------------------------------------------------------------

ClientSock::~ClientSock()
{
    delete sc;
}
//--------------------------------------------------------------------

void ClientSock::initSc(const QString &addr, const int &port)
{
    connect(sc, SIGNAL(readyRead()), this, SLOT(getData()));
    sc->connectToHost(addr, port);
    sc->waitForConnected();
}
//--------------------------------------------------------------------

QByteArray ClientSock::getData()
{
    QByteArray buf(sc->readAll());
    std::cout << m_qsScName.toStdString() << ": " << QString(buf).toStdString() << std::endl;
    return buf;
}
//--------------------------------------------------------------------

void ClientSock::setName(const QString &name)
{
    m_qsScName = name;
}
//--------------------------------------------------------------------

void ClientSock::sendData(QByteArray data)
{
    if(sc->state() == QAbstractSocket::ConnectedState)
    {
        sc->write(data);
        sc->waitForBytesWritten();
    }
}
//--------------------------------------------------------------------

void ClientSock::timerEvent(QTimerEvent *event)
{
    if(sc->state() == QAbstractSocket::UnconnectedState)
    {
        emit scDisconnect(m_qsScName);
        killTimer(event->timerId());
        std::cout << m_qsScName.toStdString() << " status: CLOSED" << std::endl;
    }
    else
    {
        //std::cout << m_qsScName.toStdString() << " status: OK" << std::endl;
    }
}
//--------------------------------------------------------------------

QString ClientSock::getName()
{
    return m_qsScName;
}
//--------------------------------------------------------------------
\end{lstlisting}

adduserdialog.h:

\begin{lstlisting}
#ifndef ADDUSERDIALOG_H
#define ADDUSERDIALOG_H

#include <QDialog>

namespace Ui {
class AddUserDialog;
}

class AddUserDialog : public QDialog
{
    Q_OBJECT

public:
    explicit AddUserDialog(QWidget *parent = 0);
    ~AddUserDialog();
    QString getClientName();
    QString getClientAddress();
    int getClientPort();

private:
    Ui::AddUserDialog *ui;
};

#endif // ADDUSERDIALOG_H
\end{lstlisting}

adduserdialog.cpp:

\begin{lstlisting}
#include "adduserdialog.h"
#include "ui_adduserdialog.h"

AddUserDialog::AddUserDialog(QWidget *parent) :
    QDialog(parent),
    ui(new Ui::AddUserDialog)
{
    ui->setupUi(this);
}

AddUserDialog::~AddUserDialog()
{
    delete ui;
}
//----------------------------------------------------------------------

QString AddUserDialog::getClientName()
{
    return ui->leClientName->text();
}
//----------------------------------------------------------------------

QString AddUserDialog::getClientAddress()
{
    return ui->leClientAddress->text();
}
//----------------------------------------------------------------------

int AddUserDialog::getClientPort()
{
    return ui->spClientPort->value();
}
\end{lstlisting}
