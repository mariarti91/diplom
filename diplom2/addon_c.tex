\ESKDappendix{Справочное}{\normalfont Исходные коды эмулятора УСПД \label{app:code_uspd}}

main.cpp:

\begin{lstlisting}
#include <QCoreApplication>
#include <iostream>

#include "clientsock.h"
#include "testserv.h"

int main(int argc, char** argv)
{
	QCoreApplication app(argc, argv);
    TestServ serv;
    serv.initServ(QString(argv[1]).toInt());
	return app.exec();
}

\end{lstlisting}

clientsock.h:

\begin{lstlisting}
#ifndef CLIENTSOCK_H
#define CLIENTSOCK_H

#include <QObject>
#include <QString>
#include <QTcpSocket>
#include <QTimer>
#include <QTimerEvent>
#include <iostream>

#include "globaldefines.h"

class ClientSock : public QObject
{
    Q_OBJECT
public:
    explicit ClientSock(QObject *parent = 0);
    explicit ClientSock(QString addr, int port, QObject *parent = 0);
    explicit ClientSock(QTcpSocket *sock, QObject *parent = 0);
    ~ClientSock();
    void initSc(const QString & addr, const int & port);
    void setName(const QString & name);
    QString getName();
    void sendData(QByteArray data);
    void timerEvent(QTimerEvent *event);
private:
    QTcpSocket * sc;
    QString m_qsScName;
signals:
    void scDisconnect(QString);
public slots:
    QByteArray getData();
};

#endif // CLIENTSOCK_H

\end{lstlisting}

globaldefines.h:

\begin{lstlisting}
#ifndef GLOBALDEFINES_H
#define GLOBALDEFINES_H

#define NORMAL_TO 6000
#define ERROR_TO 5000
#define SC_COUNT 3

#endif // GLOBALDEFINES_H

\end{lstlisting}

clientsock.cpp:

\begin{lstlisting}
#include "clientsock.h"

ClientSock::ClientSock(QObject *parent) :
    QObject(parent)
{
    sc = new QTcpSocket(this);
}
//--------------------------------------------------------------------

ClientSock::ClientSock(QString addr, int port, QObject *parent) :
    QObject(parent)
{
    sc = new QTcpSocket(this);
    initSc(addr, port);
}
//--------------------------------------------------------------------

ClientSock::ClientSock(QTcpSocket *sock, QObject *parent) :
    QObject(parent)
{
    sc = sock;
    this->setName(QString("%1").arg(sock->peerPort()));
    connect(sc, SIGNAL(readyRead()), this, SLOT(getData()));
    startTimer(5000);
}

//--------------------------------------------------------------------

ClientSock::~ClientSock()
{
    delete sc;
}
//--------------------------------------------------------------------

void ClientSock::initSc(const QString &addr, const int &port)
{
    sc->connectToHost(addr, port);
    connect(sc, SIGNAL(readyRead()), this, SLOT(getData()));
}
//--------------------------------------------------------------------

QByteArray ClientSock::getData()
{
    QByteArray buf(sc->readAll());
    std::cout << m_qsScName.toStdString() << ": " << QString(buf).toStdString() << std::endl;
    quint16 uspdId = 0x0003;
    QString uspdData("it's USPD status");
    QString uuData("it's UU data");
    QByteArray responce;
    QDataStream res(&responce, QIODevice::WriteOnly);
    //responce.append(uspdId);
    //responce.append(uspdData.length());
    //responce.append(uspdData);
    //responce.append(uuData.length());
    //responce.append(uuData);
    res << uspdId << uspdData.toUtf8() << uuData.toUtf8();
    sendData(responce);
    this->deleteLater();
    return buf;
}
//--------------------------------------------------------------------

void ClientSock::setName(const QString &name)
{
    m_qsScName = name;
}
//--------------------------------------------------------------------

void ClientSock::sendData(QByteArray data)
{
    if(sc->state() == QAbstractSocket::ConnectedState)
    {
        sc->write(data);
        sc->waitForBytesWritten();
    }
}
//--------------------------------------------------------------------

void ClientSock::timerEvent(QTimerEvent *event)
{
    if(sc->state() == QAbstractSocket::UnconnectedState)
    {
        killTimer(event->timerId());
        std::cout << m_qsScName.toStdString() << " status: CLOSED" << std::endl;
        emit scDisconnect(m_qsScName);
    }
    else
    {
        //std::cout << m_qsScName.toStdString() << " status: OK" << std::endl;
    }
}
//--------------------------------------------------------------------

QString ClientSock::getName()
{
    return m_qsScName;
}
//--------------------------------------------------------------------

\end{lstlisting}

testserv.h:

\begin{lstlisting}
#ifndef TESTSERV_H
#define TESTSERV_H

#include <QObject>
#include <QTcpServer>
#include <iostream>

#include "clientsock.h"

class TestServ : public QObject
{
    Q_OBJECT
public:
    explicit TestServ(QObject *parent = 0);
    ~TestServ();

    void initServ(const int &port);

private:
    QTcpServer *srv;
    QMap<QString, ClientSock*> users;
signals:

public slots:
    void newConnection();
    void connectClosed(QString name);
};

#endif // TESTSERV_H

\end{lstlisting}

testserv.cpp:

\begin{lstlisting}
#include "testserv.h"

TestServ::TestServ(QObject *parent) :
    QObject(parent)
{
    srv = new QTcpServer(this);
    connect(srv, SIGNAL(newConnection()), this, SLOT(newConnection()));
}
//----------------------------------------------------------

TestServ::~TestServ()
{
    delete srv;
}
//----------------------------------------------------------

void TestServ::initServ(const int &port)
{
    if(srv->listen(QHostAddress::Any, port))
    {
        std::cout << "server up!" << std::endl;
    }
    else
    {
        std::cout << "ooooooooooops!!!!!" << std::endl;
    }
}
//----------------------------------------------------------

void TestServ::newConnection()
{
    std::cout << "get connection =)" << std::endl;
    //users.push_front(srv->nextPendingConnection());
    ClientSock * sock = new ClientSock(srv->nextPendingConnection());
    connect(sock, SIGNAL(scDisconnect(QString)), this, SLOT(connectClosed(QString)));
    users.insert(sock->getName(), sock);
}
//----------------------------------------------------------

void TestServ::connectClosed(QString name)
{
    delete users.value(name);
    users.remove(name);
}

\end{lstlisting}
