\newpage
\section{Технико-экономическое обоснование}
\setcounter{figure}{0}

\subsection{Обоснование необходимости проводимого исследования}

В настоящее время автоматизированные системы коммерческого учета энергоресурсов(АСКУЭ) широко применяются на предприятиях. Это простой и эффективный способ контроля потребляемых предприятием ресурсов ресурсов. В настоящее время проводятся работы по реализации АСКУЭ, предназначенных для городских электросетей.

Изучая вопрос информационной безопасности таких систем, стало ясно
что существующие решения не подходят для использования, так как масштабы систем на предприятиях меньше и нет необходимости защищать систему от НСД, так как она находится в охраняемой зоне.

Цель данной работы — создать безопасный протокол взаимодействия устройства сбора и передачи данных (УСПД) с сервером сбора данных(ССД). На данном участке существующие протоколы плохо применимы ввиду больших расстояний и отсутствия защитных механизмов.

Для решения данной проблемы принято решение разработать новый протокол взаимодействия УСПД с ССД.

\subsection{Планирование комплекса работ по разработке программного обеспечения}

Основными задачами планирования работ являются:

\begin{itemize}
 \item определение объема предстоящих работ;
 \item взаимная увязка работы и установление рациональной последовательности предстоящих работ;
 \item установление сроков выполнения работ;
 \item определение необходимых, для выполнения планируемых работ денежных, материальных и трудовых ресурсов.
\end{itemize}

При выполнении дипломной работы было задействовано два человека:

\begin{itemize}
 \item руководитель;
 \item разработчик.
\end{itemize}

Руководитель выполняет контроль выполнения различных этапов работ, согласованность этапов выполнения работ между собой, корректирует действия разработчика, дает рекомендации по выполнению тех или иных работ. Разработчик реализует тот объем работ, который установлен руководителем в соответствие с техническим заданием.

Месячный оклад техника в ТУСУР составляет 3164 рубля, с учетом 20 рабочих дней в месяце, и 8 часового рабочего дня, стоимость одного часа работ равна 19,77 рублей. Месячный оклад руководителя к.н., доцента в университете равен 14 700 рублей, с учетом 24 рабочих дней, и 6 часового рабочего дня, стоимость одного часа работ равна 102,08 рубля.

ТУТ КОРОЧЕ ТАБЛИЧКО СО ССЫЛКОЙ НА НЕЁ =) %TODO{1.1}

Зная длительность цикла каждого этапа и возможность их параллельно-последовательного выполнения, можно рассчитать срок завершения планируемых работ и составить ленточный и сетевой графики плана их выполнения. Поскольку работа не требует большого состава исполнителей, то ограничимся ленточным графиком планирования, представленным в приложении А.

\subsection{Определение сметной стоимости проекта}

\subsubsection{Общие положения}

Смета затрат для данной работы состоит из расходов, которые включают в себя следующие статьи:

\begin{itemize}
 \item затраты на оборудование и амортизацию;
 \item расходы на оплату труда и отчисления на социальные нужды;
 \item затраты на основные и вспомогательные материалы;
 \item затраты на электроэнергию.
\end{itemize}

\subsubsection{Затраты на оборудование и амортизацию}

Основным оборудованием при проведении работы являются компьютер и принтер, которые постановлением Правительства Российской Федерации от 1.01.02 г. N 1 отнесены ко второй амортизационной группе – «имущество со сроком полезного использования свыше 2 лет до 3 лет включительно». Месячная норма амортизации составляет 2,8\% и для компьютера, и для принтера.

ТУТ КОРОЧЕ ОПЯТЬ ТАБЛИЧКО %TODO{1.3}

\subsubsection{Расходы на оплату труда и отчисления на социальные нужды}

Статья затрат учитывает выплаты по заработной плате за выполненную работу, исчисленные на основании тарифных ставок и должностных окладов в соответствии с принятой в организации-разработчике системой оплаты труда. В этой статье также отражаются премии, надбавки и доплаты за условия труда, оплата ежегодных отпусков, выплата районного коэффициента и некоторые другие расходы. Отчисления на социальные нужды учитывают страховые взносы.

ТУТ КОРОЧЕ ОПЯТЬ ТАБЛИЧКО %TODO{1.4}

\subsubsection{Затраты на основные и вспомогательные материалы}

Статья включает расходы по приобретению и доставке основных и вспомогательных материалов, необходимых для опытно-экспериментальной проработки решения, для изготовления макета или опытного оборудования. Сюда включаются и стоимость необходимых материалов для изготовления образцов и макетов, и материалов необходимых для оформления требуемой документации. 

Размер транспортно-заготовительных расходов (ТЗР), определяемый в процентах от стоимости, примем 10\%. Стоимость вспомогательных материалов принимается 10\% от стоимости основных материалов с учетом ТЗР. 

ТУТ КОРОЧЕ ОПЯТЬ ТАБЛИЧКО %TODO{1.5}

\subsubsection{Расходы на электроэнергию}

Статья включает затраты по электроэнергии на технологические нужды. В настоящее время тариф на электроэнергию для населения г. Томска на 2015 год составляет 2,7 руб./ кВт ч. Введенный приказом от 26.12.2014 " №6/9 (691) «О тарифах на электрическую энергию для населения и потребителей, приравненных к категории население по Томской области на 2015 год», принятый департаментом тарифного регулирования Томской области.

ТУТ КОРОЧЕ ОПЯТЬ ТАБЛИЧКО %TODO{1.6}

\subsubsection{Накладные расходы}

ТУТ КОРОЧЕ ОПЯТЬ ТАБЛИЧКО %TODO{1.7}

\subsubsection{Сводная смета затрат}

ТУТ КОРОЧЕ ОПЯТЬ ТАБЛИЧКО %TODO{1.8}

\subsection{Экономический эффект}

Рассмотрим показатели системы, смена работника длится 8 часов, за вычетом предоставляемого перерыва время работы составляет 420 минут. Это время оператор тратит на сбор показаний. Сбор показаний производится вручную. Если предположить что сбор показаний с одного этажа на 4 квартиры занимает 10 минут, то за смену оператор сможет собрать показания с 42 этажей, для простоты расчетов примем число этажей равное 40, то есть мы собираем данные с 10 этажного 4 подъездного дома. Тогда за одну смену оператор сможет собрать показания со 160 квартир. Если же собирать данные будет автоматизированная система, то данные с подъезда будут собираться за один цикл опроса УСПД, которые занимает до 5 минут. Тогда система опросит наш дом за 20 минут, а за смену она обработает 21 такой дом. Сейчас для опроса 21 дома нам понадобится 21 оператор. После внедрения системы, каждая квартира будет опрашиваться на 2,5 минуты быстрее, значит на опросе 3360 квартир мы экономим 8400 минут или 140 часов, что составляет 17,5 8и-часовых рабочих смен, то есть нам понадобится на 17 операторов меньше, а значит для обслуживания 3360 квартир нам понадобится 4 операторов.

ТУТ КОРОЧЕ ОПЯТЬ ТАБЛИЧКО %TODO{1.9}

БЛАБЛАБЛА

ТУТ КОРОЧЕ ОПЯТЬ ТАБЛИЧКО %TODO{1.10}

Из таблицы видно, что прогнозный экономический эффект на данном этапе разработки защищенного протокола взаимодействия УСПД с ССД составляет 2,8 миллиона рублей в год при условии что система обслуживает 21 десятиэтажный 4 подъездный дом, или 3360 устройств учета. Данный показатель свидетельствует о целесообразности продолжения проекта ввиду того, что на данный момент себестоимость проекта составляет менее 45 тысяч рублей.