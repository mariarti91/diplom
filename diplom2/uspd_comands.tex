\newpage
\section{Система команд УСПД}
\setcounter{table}{0}

УСПД должен отвечать за сбор данных с подконтрольных ему счетчиков, настройку этих счетчиков, контроль их состояния, должен давать возможность обращаться к счетчику с отладочными командами, так же на самом УСПД будет реализован контроль доступа. Исходя из всего этого, можно обозначить типы команд, необходимые нам для работы с УСПД.

Для взаимодействия с УСПД понадобятся следующие типы команд: 

\begin{itemize}
 \item регистрационные;
 \item настройка;
 \item запроса данных;
 \item запроса статуса;
 \item отладочные;
 \item управления.
\end{itemize}

\subsection{Команды регистрации на УСПД}

Данные команды отвечают за процедуры идентификации и аутентификации пользователя на УСПД. Перечень команд представлен в таблице \ref{tab:ident_comand}.

% \begin{center}
\begin{table}[!ht]
 \caption{Команды идентификации} 
 \label{tab:ident_comand}
 \begin{tabular}{|*3{p{5cm}|}}
  \hline
  Команда & описание & формат \\
  \hline
  регистрация в системе & передача имени пользователя и начало процедуры авторизации & login user\_name \\
  \hline
  завершение сеанса & завершение работы пользователя и выход из системы & logout \\
  \hline
 \end{tabular}
% \end{center}
\end{table}
\subsection{Команды настройки УСПД}

Данные команды предназначены для настройки УСПД и подконтрольных ему счетчиков. Присвоения идентификаторов, установки интервала опроса счетчиков, разграничения доступа к функционалу. Перечень команд представлен в таблице \ref{tab:config_comand}.

\begin{table}[!ht]
 \caption{Команды настройки УСПД} 
 \label{tab:config_comand}
 \begin{tabular}{|*3{p{5cm}|}}
  \hline
  Команда & описание & формат \\
  \hline
  конфигурация системы & производится поиск подключенных датчиков, присвоение им id и вывод информации о проделанной работе & configure \\
  \hline
  установка времени & устанавливает системное время & set\_time value \\
  \hline
 \end{tabular}
\end{table}

\subsection{Команды получения данных от УСПД}

Данные команды используются для получения информации от счетчика или их группы. Перечень команд представлен в таблице \ref{tab:gd_comand}. 

\begin{table}[!ht]
 \caption{Команды управления дынными}
 \label{tab:gd_comand} 
 \begin{tabular}{|*3{p{5cm}|}}
  \hline
  Команда & описание & формат \\
  \hline
  получение данных от счетчиков & отправляет запрос данных с указанного счетчика & get\_data device\_id \\
  \hline
 \end{tabular}
\end{table}

\subsection{Команды получения статуса УСПД}

Данные команды используются для получения информации о состоянии УСПД и подконтрольных ему счетчиков. Перечень команд представлен в таблице \ref{tab:gs_comand}.

\begin{table}[!ht]
 \caption{Команды получения системной информации}
 \label{tab:gs_comand}  
 \begin{tabular}{|*3{p{5cm}|}}
  \hline
  Команда & описание & формат \\
  \hline
  получение информации о состоянии системы & Получение значения времени, количества счетчиков, времени последнего сеанса передачи даных & get\_sys\_info \\
  \hline
  получение списка идентификаторов счетчиков & Возвращает список device\_id всех подконтрольных счетчиков & get\_devices\_list \\
  \hline
  получить сведения о состоянии счетчика & получить статус указанного счетчика & get\_sensore\_status device\_id \\
  \hline
  системное время & возвращает текущее значение времени на УСПД & get\_time \\
  \hline
 \end{tabular}
\end{table}

\newpage
\subsection{Команды отладки УСПД}

Данные команды предназначены для разработчиков и помогают отлаживать систему. Перечень команд представлен в таблице \ref{tab:debug_comand}.

\begin{table}[!ht]
 \caption{Отладочные команды}
 \label{tab:debug_comand} 
 \begin{tabular}{|*3{p{5cm}|}}
  \hline
  Команда & описание & формат \\
  \hline
  установить значения счетчика & указывает, какие данные должен вернуть счетчик при следующем его опросе & set\_data device\_id data \\
  \hline
 \end{tabular}
\end{table}

\subsection{Команды управления УСПД}

Команды перезагрузки, сброса настроек и прочее. Перечень команд представлен в таблице \ref{tab:sys_comand}.

\newpage
\begin{table}[!ht]
 \caption{Системные команды}
 \label{tab:sys_comand} 
 \begin{tabular}{|*3{p{5cm}|}}
  \hline
  Команда & описание & формат \\
  \hline
  перезагрузка устройства & перезагружает систему & restart \\
  \hline
  обнуление показаний & сбрасывает значения указанного счетчика & renew device\_id \\
  \hline
 \end{tabular}
\end{table}

Одним их условий разработки УСПД является его максимальная простота\cite{tz}, поэтому в системе команд допускаются только команды непосредственной настройки УСПД или обращения к конкретному счетчику. Более сложные возможности (такие как опрос определенной группы счетчиков или всех счетчиков) будут реализованы на серверной стороне. То есть будет использоваться многоуровневая система драйверов.

