\section{Используемые протоколы}

На рынке представлено множество датчиков, которые связываються с системой по различным протоколам. Используются такие протоколы как [2]:
\begin{itemize}
\item TCP/IP;
\item Modbus;
\item Canbus;
\item МЭК 60870-5-101;
\item МЭК 60870-5-104;
\item ОРС;
\item "Пирамида" (закрытый протокол);
\item спецпротоколы электросчетчиков;
\item SNMP;
\item TFTP;
\item RTU-325.
\end{itemize}

\subsection{Стандарт OPC}

OPC (OLE for Process Control) – семейство протоколов,предоставляющих единый интерфейс для управления объектами автоматизации и технологическими процессами. В частности, OPC DA (DataAccess) — стандарт, описывающий набор функций обмена данными в реальном времени с объектами автоматизации. 

Спецификация OPC охватывает:
\begin{itemize}
 \item концепцию клиент-серверную технологию OPC и определение данных;
 \item описания интерфейсов, методов, параметров и возможное ихповедение;
 \item описание типов и структур данных;
 \item общие виды деятельности, которые включают в себя определениеадресного пространства и егопросмотра. Чтение, запись и подписка науведомления об обновлении данных.
\end{itemize}

К недостаткам этого OPC можно отнести:
\begin{itemize}
 \item необходимость платного членствавсообществе The InteroperabilityStandard for Industrial Automation;
 \item протокол основан на технологиях windows.
\end{itemize}

\subsection{Стандарт МЭК-60870-5-101/104}

МЭК-60870-5-101/104 – это протокол передачи данных, применяется в АСКТУ, АСКУЭ. Особенности реализации протокола МЭК 60870-101/104 при передаче данных между объектом и диспетчерским центром:

\begin{itemize}
 \item передача ограниченного количества информации, что обусловлено необходимостью переназначения всех сигналов с одного протокола на другой, и, как следствие, потеря некоторых данных, передача которых на этапе проектирования не была сочтена целесообразной;
 \item отсутствие единых наименований сигналов в рамках объекта и в центрах управления сетями, приводящее к сложности наладки и отслеживания ошибок;
 \item протокол МЭК 60870-5-101 предназначен для передачи данных по последовательным линиям связи RS-232/485;
 \item протокол МЭК 60870-5-104 является расширением протокола 101 и регламентирует использование сетевого доступа по протоколу TCP/IP.
\end{itemize}

Недостатки реалищации данного стандарта:

\begin{itemize}
 \item количество передаваемых сигналов ограничивается определенным количеством дискретных входов и выходов;
 \item отсутствует возможность контроля связи между устройствами;
 \item возможно ложное срабатывание дискретного входа устройства при замыкании на землю в цепи передачи сигнала;
 \item цепи подвержены воздействию электромагнитных помех;
 \item сложность расширения систем;
 \item передача данных осуществляется в два этапа:
 \begin{enumerate}
  \item назначение индексированных коммуникационных объектов на прикладные объекты;
  \item  назначение прикладных объектов на переменные в прикладной базе данных или программе. Таким образом, отсутствует семантическая связь (полностью или частично) между передаваемыми данными и объектами данных прикладных функций.
 \end{enumerate}
 \item протоколы не предусматривают возможность передачи сигналов реального времени.
\end{itemize}

\subsection{Стандарт MODBUS}

Modbus – один из наиболее распространенных сетевых протоколов для интеграции устройств РЗА (релейная защита автоматики) в систему АСТУ, построенный на архитектуре «клиент–сервер». Данный протокол является открытым, что частично обуславливает его популярность. Протокол Modbus для передачи данных использует такие линиям связи как RS-485, RS-433, RS-232, а также сети TCP/IP (Modbus TCP).

Стандарт Modbus содержит в себе:

\begin{itemize}
 \item спецификация прикладного уровня;
 \item спецификация канального уровня;
 \item спецификация физического уровня;
 \item спецификацию ADU для транспорта через стек TCP/IP.
\end{itemize}

К достоинствам стандарта относится:

\begin{itemize}
 \item массовость;
 \item относительная простота реализации систем на его базе.
\end{itemize}

К недостаткам данного протокола можно отнести:

\begin{itemize}
 \item в случае необходимости отсутствует возможность оперативной сигнализации от конечного устройства к мастеру;
 \item стандарт не регламентирует начальную инициализацию системы. Назначение сетевых адресов и параметров системы для каждого устройства выполняются вручную на этапе адаптации;
 \item отсутствие  возможности  конечным  устройствам  обмениваться фиксированными данными друг с другом без участия мастера. Что ограничивает  применимость MODBUS-решений в системах регулирования реального времени.
\end{itemize}

\subsection{Стандарт CAN}

CAN (Controller Area Network) – стандарт промышленной сети, использующийся для объединения в единую сеть устройств и датчиков. Применяется в системах автоматизации промышленного производства. 

В первую очередь данный стандарт описывает физический уровень, наибольшую популярность получил вариант описанный в ISO 11898-2. Физический уровень использует дифференциальную передачу данных по витой паре, для управления доступом к шине используется неразрушающее bit-wise разрешение конфликтов ISO 11898.

К положительным аспектам данной реализации относиться:

\begin{itemize}
 \item работа в режиме жёсткого реального времени;
 \item высокая устойчивость к помехам;
 \item надёжный контроль ошибок приема-передачи данных.
\end{itemize}

К достоинствам данного стандарта относятся:

\begin{itemize}
 \item сообщения имеют малые размеры (8 байт данных) и защищены контрольной суммой;
 \item большой размер служебных данных в пакете по отношению к полезным данным;
 \item отсутствие единого общепринятого стандарта на протокол высокого уровня.
\end{itemize}

\subsection{Стандарт DLMS}

DLMS (Device Language Message Specification) – стек протоколов, ориентированный на 16-разрядные микроконтроллеры Microchip PIC. Является международным стандартом для систем сбора данных с электро-, газовых, водяных и тепловых счетчиков. Работает на основе протоколов связи (RS232, RS485, PSTN, GSM, GPRS, IPv4, PPP и PLC), с поддержкой шифрования AES128.

Особенности данного протокола:

\begin{itemize}
 \item работает на всех 16-битных микроконтроллерах PIC и dsPIC;
 \item возможность интеграции стека DLMS с текущими реализациями протоколов TCP/IP, ZigBee и PLC;
 \item небольшой объем занимаемой памяти позволяет использовать компактные и недорогие контроллеры;
 \item для европейского рынка стек поддерживает IEC 62056-21 Mode E.
\end{itemize}

К недостаткам данного протокола можно отнести то, что описание протокола предоставляется на платной основе или требуется членство в ассоциации DLMS.

Из представленного краткого анализа видно, что существующие протоколы связи достаточно успешно позволяют реализовывать задачи диспетчерского управления / интеграции данных в системы управления, однако не все они позволяют реализовывать функции реального времени. К тому же большое количество проприетарных протоколов приводит к усложнению процесса интеграции устройств в единую систему: 
\begin{itemize}
 \item протоколы должны поддерживаться контроллером и ЦППС, что требует реализации поддержки большого количества протоколов в УСО и ЦППС одновременно и ведет к удорожанию оборудования;
 \item для интеграции устройств по проприетарным протоколам требуется квалификация наладочного персонала в работе с каждым из них;
 \item переназначение сигналов из проприетарных протоколов в общепромышленные и назад часто приводит к потере информации, включая дополнительную информацию;
 \item при передаче данных по-прежнему применяется большое количество последовательных интерфейсов, что накладывает ограничения на скорость передачи данных, объем передаваемых данных и количество устройств, одновременно включенных в информационную сеть.
\end{itemize}

Большинство приборов учета (или УСПД) поддерживают протокол передачи данных TCP/IP, в виду чего использование данного протокола является целесообразным, так как в отличие от других протоколов передачи данных TCP/IP имеет следующие преимущества:

\begin{itemize}
 \item скорость разработки и цена разработки;
 \item более простой (простая поддержка);
 \item обеспечивает контроль целостности передаваемых данных;
 \item возможность обеспечить программную поддержку новых устройств через модульную систему драйверов.
\end{itemize}